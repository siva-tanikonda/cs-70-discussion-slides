\documentclass{beamer}
\usepackage{cancel, algpseudocode, hyperref, tikz, venndiagram, centernot}

\title{CS 70 Discussion 7B}
\date{October 18, 2024}

\begin{document}

\frame{\titlepage}

\begin{frame}[fragile]
    \frametitle{Halting Problem}
    {\bf Problem}: Given any program \texttt{P} and any input \texttt{x}, can you create a program \texttt{TestHalt(P,x)} that checks if \texttt{P(x)} halts?\\
    {\bf Solution}: No. Consider designing \texttt{P}:
    \begin{verbatim}
    function P(x):
        if TestHalt(P,x) returns true:
            loop forever
        else:
            return nothing
    \end{verbatim}
    \texttt{TestHalt} can never work on our \texttt{P} (in fact, \texttt{P} is a non-executable program), so a correct implementation of \texttt{TestHalt} can never exist ({\bf self-reference} proof!).
\end{frame}

\begin{frame}
    \frametitle{Computability}
    {\bf Problem}: How do we prove that a problem $X$ can be solved?\\
    {\bf Solution}: Just write a program \texttt{P} that can solve your problem!
\end{frame}

\begin{frame}
    \frametitle{Reductions}
    {\bf Problem}: How do we prove a problem $X$ can't be solved?
    {\bf Solution}: Prove that being able to solve $X$ implies that I can solve an unsolvable problem. In this class, we often establish the implication:
    \begin{gather*}
        \text{$X$ can be solved}\implies\text{Halting Problem can be solved}
    \end{gather*}
    Since we know (ground-truth) that the Halting Problem has no solution, if the above implication is true, then we have no choice but to conclude $X$ can't be solved. Therefore, the Halting Problem {\bf reduces} to problem $X$ (i.e. solving the Halting Problem is at most as hard as solving problem $X$).
\end{frame}

\end{document}
