\documentclass{beamer}
\usepackage{cancel, algpseudocode, hyperref, tikz, venndiagram, centernot}

\title{CS 70 Discussion 3B}
\date{September 20, 2024}

\begin{document}

\frame{\titlepage}

\begin{frame}
    \frametitle{Modulo Operation}
    {\bf Basic Definition}: $a\bmod m=\text{remainder of }a\text{ divided by }m$ (ex. $14\bmod 5=4$)\\
    {\bf Residue Classes}: $a\equiv b\pmod m$ means $(\exists k\in\mathbb{Z})(a=b+km)$ (i.e. $b-a$ is a multiple of $m$)
    \begin{itemize}
        \item In this case, we say that $a$ and $b$ are in the same ``residue class'' modulo $m$
    \end{itemize}
    Some useful formulas to note:
    \begin{itemize}
        \item $a+b\equiv (a\bmod m)+(b\bmod m)\pmod m$
        \item $a-b\equiv (a\bmod m)-(b\bmod m)\pmod m$
        \item $a\times b\equiv (a\bmod m)\times (b\bmod m)\pmod m$
        \item $a^b\equiv (a\bmod m)^b\pmod m$
    \end{itemize}
\end{frame}

\begin{frame}
    \frametitle{Euclid's Algorithm}
    {\bf Problem}: How do we easily get the greatest-common divisor (the largest integer that divides two numbers $a$ and $b$) of two numbers?\\
    {\bf Algorithm}:
    \begin{gather*}
    \gcd(a,b)=\begin{cases}\gcd(b,a\bmod b)&\text{if }b>0\\a&\text{else}\end{cases}
    \end{gather*}
    Example:
    \begin{align*}
    &\gcd(24,42)=\gcd(42,24\bmod 42)&\\
    &\phantom{\gcd(24,42)}=\gcd(42,24)&\\
    &\phantom{\gcd(24,42)}=\gcd(24,42\bmod 24)&\\
    &\phantom{\gcd(24,42)}=\gcd(24,18)&\\
    &\phantom{\gcd(24,42)}=\gcd(18,24\bmod 18)&\\
    &\phantom{\gcd(24,42)}=\gcd(18,6)&\\
    &\phantom{\gcd(24,42)}=\gcd(6,18\bmod 6)&\\
    &\phantom{\gcd(24,42)}=\gcd(6,0)=6&
    \end{align*}
\end{frame}

\begin{frame}
    \frametitle{Inverses}
    An inverse of an integer $a$ in modspace $m$ is another integer $a^{-1}$ such that:
    \begin{gather*}
        a\times a^{-1}\equiv 1\pmod m
    \end{gather*}
    Example: Inverse of $2$ mod $5$ is $3$ (i.e. $2^{-1}\equiv 3\pmod 5$):
    \begin{gather*}
        2(3)\equiv 6\equiv 1\pmod 5
    \end{gather*}
    $a\pmod m$ has an inverse iff (if and only if) $\gcd(a,m)=1$ (i.e. $a$ and $m$ are {\bf coprime}). Multiplying by modular inverses is the way to emulate ``division'' in modspace.
\end{frame}

\end{document}