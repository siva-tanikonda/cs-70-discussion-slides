\documentclass{beamer}
\usepackage{cancel, algpseudocode, hyperref, tikz, venndiagram, centernot}

\title{CS 70 Discussion 9A}
\date{October 30, 2024}

\begin{document}

\frame{\titlepage}

\begin{frame}
    \frametitle{Union Bound}
    {\bf Problem}: For a set of events $A_1,A_2,\dots,A_n$, how do we {\it upper-bound} $\mathbb{P}\left[\bigcup_{i=1}^n A_i\right]$ without having to use the Principle of Inclusion-Exclusion?\\
    {\bf Solution}: We use the {\bf union-bound}, which states that in all cases:
    \begin{gather*}
        \mathbb{P}\left[\bigcup_{i=1}^nA_i\right]\leq\sum_{i=1}^n\mathbb{P}[A_i]
    \end{gather*}
\end{frame}

\begin{frame}
    \frametitle{Symmetry}
    {\bf Problem}: Consider a discrete sample space $\Omega$. We have an event $A$ for which we want to find $\mathbb{P}[A]$. How do we use the probability $\mathbb{P}[B]$ of some other event $B$ to help us?\\
    {\bf Solution}: Find a {\it bijective} function $f:A\rightarrow B$ that also satisfies the property:
    \begin{gather*}
        (\forall\omega\in A)(\mathbb{P}[\omega]=\mathbb{P}[f(\omega)])
    \end{gather*}
    If we manage to find such a function, we can then conclude that $\mathbb{P}[A]=\mathbb{P}[B]$.
\end{frame}

\end{document}
