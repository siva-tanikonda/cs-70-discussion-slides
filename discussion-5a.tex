\documentclass{beamer}
\usepackage{cancel, algpseudocode, hyperref, tikz, venndiagram, centernot}

\title{CS 70 Discussion 5A}
\date{October 2, 2024}

\begin{document}

\frame{\titlepage}

\begin{frame}
    \frametitle{Polynomials}
    {\bf Coefficient Representation}: $n$ values $c_0,c_1,...,c_{n-1}\in\mathbb{R}$ define a {\it unique} degree $\leq n-1$ polynomial:
    \begin{gather*}
        f(x)=c_0+c_1x+c_2x^2+\dots+c_{n-1}x^{n-1}
    \end{gather*}
    {\bf Point Representation}: $n$ {\it distinct} points $(a_0,b_0),(a_1,b_1),\dots,(a_{n-1},b_{n-1})\in\mathbb{R}^2$ define a {\it unique} degree $\leq n-1$ polynomial.
\end{frame}

\begin{frame}
    \frametitle{Lagrange Interpolation}
    {\bf Goal}: Given $n$ distinct points $(a_0,b_0),(a_1,b_1),...,(a_{n-1},b_{n-1})\in\mathbb{R}^2$, how do we generate our unique degree $\leq n-1$ polynomial $f(x)$?\\
    {\bf Solution}:
    \begin{gather*}
        f(x)=\sum_{i=0}^{n-1}b_i\Delta_i(x)\\
        \Delta_i(x)=\frac{\prod_{j\neq i}(x-a_j)}{\prod_{j\neq i}(a_i-a_j)}
    \end{gather*}
    $\Delta_i(x)$ is often referred-to as a {\it delta polynomial}. Also, note our construction for a polynomial is similar to the construction of the unique solution $x$ for CRT.
\end{frame}

\begin{frame}
    \frametitle{Secret Sharing}
    {\bf Goal}: We have a secret. We want a group of $n$ people to only be able to learn the secret if at least $m$ people agree to unlock the secret.\\
    {\bf Solution}:
    \begin{enumerate}
        \item Generate a random polynomial of degree $m-1$ where $f(0)$ is our secret.
        \item Pick $n$ distinct points on our polynomial where each $x$-coordinate is $>0$.
        \item Give each person a unique one of the $n$ points.
        \item Two cases: \begin{itemize}
            \item $\geq m$ people agree: $m$ points uniquely define a degree $\leq m-1$ polynomial, so construct $f(x)$ with Lagrange interpolation and get the secret at $f(0)$.
            \item $<m$ people agree: We need at least $m$ points to retrieve our polynomial, or else we still can't narrow-down to exactly $1$ unique polynomial (i.e. we still have infinite possibilities for $f(0)$).
        \end{itemize}
    \end{enumerate}
\end{frame}

\begin{frame}
    \frametitle{Galois Field}
    {\bf Problem}: We have numerical instability with all our arithmetic operations during Lagrange interpolation (floating-point operations are imprecise on computers).\\
    {\bf Solution}: We use modular arithmetic. We define our polynomial within $\text{GF}(p)$, where $p$ is some large prime. Simply, this means that every arithmetic operation in this space is done $\bmod p$:
    \begin{itemize}
        \item $a+b=c\rightarrow a+b\equiv d\pmod p$
        \item $a-b=c\rightarrow a-b\equiv d\pmod p$
        \item $ab=c\rightarrow ab\equiv d\pmod p$
        \item $\frac{a}{b}=c\rightarrow ab^{-1}\equiv d\pmod p$
        \begin{itemize}
            \item $b^{-1}$ is modular inverse of $b\pmod p$
        \end{itemize}
    \end{itemize}
    Now, we don't need to worry about floating-point numbers. Also, Lagrange interpolation works fine in $\text{GF}(p)$, as we only perform additions, subtractions, multiplications, and divisions to construct our polynomial.
\end{frame}

\end{document}